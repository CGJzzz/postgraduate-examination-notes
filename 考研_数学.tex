%% LyX 2.2.3 created this file.  For more info, see http://www.lyx.org/.
%% Do not edit unless you really know what you are doing.
\documentclass[UTF8]{ctexart}
\usepackage{amsmath}
\usepackage{amssymb}
\usepackage{fontspec}
\usepackage[unicode=true,pdfusetitle,
 bookmarks=true,bookmarksnumbered=true,bookmarksopen=true,bookmarksopenlevel=1,
 breaklinks=false,pdfborder={0 0 1},backref=false,colorlinks=false]
 {hyperref}

\makeatletter
%%%%%%%%%%%%%%%%%%%%%%%%%%%%%% User specified LaTeX commands.
% 重定义\nobreakspace命令,避免XeTeX出错(2.1.2版本后已不需要)
\DeclareRobustCommand\nobreakspace{\leavevmode\nobreak\ }

% 解决pdflatex不支持插入eps图片问题
\usepackage{epstopdf}
\usepackage[left=2cm, right=2cm, top=2cm]{geometry}

\makeatother

\usepackage{xunicode}
\begin{document}
1.当题目中有关于积分的不等式时,可以考虑构造一个原函数$F(x)$使$F^{\prime}(x)$有不等式的形式,如$F(x)=x\int_{x}^{1}f(y)dy$或直接求它的导函数$f^{\prime}(x)$;使用积分中值定理{[}与积分面积相等的矩形必与曲线相交于一点$(\xi,f(\xi))${]}

2.当题目中有等式或函数某点值时,可考虑拉格朗日中值定理和罗尔定理.(拉格朗日中值定理与一阶泰勒公式等价.)

3.题目中有定积分时可考虑将常数项凑成定积分.

4.注意定义域.

5.$(\ln(x+\sqrt{x^{2}\pm1}))^{\prime}=\frac{1}{\sqrt{x^{2}\pm1}}$,$(\arcsin x)^{\prime}=\frac{1}{\sqrt{1-x^{2}}}$,$(\frac{1}{2}x\sqrt{x^{2}\pm1}\pm\frac{1}{2}\ln|x+\sqrt{x^{2}\pm1}|)^{\prime}=\sqrt{x^{2}\pm1}$,$(\frac{1}{2}x\sqrt{1-x^{2}}+\frac{1}{2}\arcsin x)^{\prime}=\sqrt{1-x^{2}}$(可用dx中的x分部积分得出)

6.分部积分时设积分结果为$I$,要尽量凑出$I$以构成关于$I$的方程,如$I=\int_{0}^{\frac{m\pi}{2}}\sin^{n}xdx$要尽量使右侧出现$\sin x$.

7.能用三角解决的不要用无理式,有$\sin^{2}x$,$\cos^{2}x$可凑$\tan^{2}x$,有$\frac{1}{\cos^{2}x}dx$可凑$d(\tan x)$,有$\sqrt{1-x^{2}}$可换$\sin t$,想办法凑$\tan t$.

8.$\int\frac{du}{(\frac{u}{2})^{2}+1}\neq\arctan\frac{u}{2}$不要忘记积分变量与变量一致性.

9.积分结果一定要求导验证.

10.能拆出常数项或多项式项的要先拆后积分,如$\int\frac{du}{(2u-1)(u^{2}+1)}=\int\frac{u^{2}+1-u^{2}}{(2u-1)(u^{2}+1)}du$从高阶开始配凑,也可使用待定参数,令$\frac{1}{(2u-1)(u^{2}+1)}=A\frac{1}{2u-1}+B\frac{u}{u^{2}+1}+C\frac{1}{u^{2}+1}$解得$A=\frac{4}{5},B=-\frac{2}{5},C=-\frac{1}{5}$.归纳:分母要化为$(A_{1}x+B_{1})^{n}(A_{2}x^{2}+B_{2})(A_{3}x^{2}+B_{3}x+C_{3})$时,若后两式不能因式分解则最简分量必有$\frac{1}{(A_{1}x+B_{1})^{n}},\frac{x}{A_{2}x^{2}+B_{2}},$$\frac{1}{A_{2}x^{2}+B_{2}},$

$\frac{x}{A_{3}x^{2}+B_{3}x+C_{3}},\frac{1}{A_{3}x^{2}+B_{3}x+C_{3}}$.

11.柯西不等式$[\int_{a}^{b}f(x)g(x)dx]^{2}\leq\int_{a}^{b}[f(x)]^{2}dx\cdot\int_{a}^{b}[g(x)]^{2}dx$.

12.直线$l$$\frac{x-x_{0}}{a}=\frac{y-y_{0}}{b}=\frac{z-z_{0}}{c}$方向向量为$\vec{l}=(a,b,c)$,经过点$L(x_{0},y_{0},z_{0})$,点$M$到$l$的距离为$d=\frac{|\vec{l}\times\vec{L}|}{|\vec{l}|}$.

13.一条直线与另外两条直线相交,它们的方向向量没有直接关系,也不能联立两直线求交点,认为它在第三条直线上,因为交点可能有两个.

14.若$f(x,y)$在$(x,y)$点可微,则$f(x+\Delta x,y+\Delta y)-f(x,y)=f'_{x}(x,y)\Delta x+f'_{y}(x,y)\Delta y=O(\sqrt{(\Delta x)^{2}+(\Delta y)^{2}})$.

15.$\frac{d}{dy}f(0,y)|_{y=0}=\lim\limits_{\Delta y\to0}\frac{f(0,\Delta y)-f(0,0)}{\Delta y}=\lim\limits_{\Delta y\to0}[\frac{f(x,\Delta y)-f(x,0)}{\Delta y}]_{x=0}=\frac{\partial}{\partial y}f(x,y)|_{(0,0)}$,因此可在求偏导前对无关变量赋值.

16.设$\vec{l}$与$x,y,z$轴夹角分别为$\alpha,\beta,\gamma$,则函数$u(x,y,z)$沿$\vec{l}$方向的方向导数为$\frac{\partial u}{\partial\vec{l}}=\frac{\partial u}{\partial x}\cos\alpha+\frac{\partial u}{\partial y}\cos\beta+\frac{\partial u}{\partial z}\cos\gamma$.

17.$[f(u(x,y),v(x,y))]_{x}^{\prime}=f_{1}^{\prime}u_{x}^{\prime}+f_{2}^{\prime}v_{x}^{\prime}.$

18.$u(x,y,z)$在$\nabla u$方向上的方向导数最大,值为$|\nabla u|$.

19.$f_{x}^{\prime}(x,y,z)$ $(z=z(x,y))$ $\neq$ $\frac{\partial}{\partial x}[f(x,y,z(x,y))]$.

20.$\varphi_{x}^{\prime}(2x^{2})=\varphi^{\prime}(2x^{2})\cdot4x$.

21.当$f_{12}^{\prime\prime}=f_{21}^{\prime\prime}$时,最后结果应合并.

22.隐函数求导之前应将所有$z=z(x,y)$标记出来以免遗忘.

23.多项式要统一格式,如$x-y,y-x$应统一为$x-y,-(x-y)$.

24.求导要一步一步地求,不要跳步.

25.对$f(x,y)=0$可用全微分为0得到$\frac{dy}{dx}=-\frac{f_{x}^{\prime}}{f_{y}^{\prime}}$.

26.$\frac{\partial(x,y)}{\partial(u,v)}=\begin{vmatrix}\frac{\partial x}{\partial u} & \frac{\partial x}{\partial v}\\
\frac{\partial y}{\partial u} & \frac{\partial y}{\partial v}
\end{vmatrix}=\begin{vmatrix}\begin{bmatrix}\frac{\partial x}{}\\
\frac{\partial y}{}
\end{bmatrix}\begin{bmatrix}\frac{}{\partial u} & \frac{}{\partial v}\end{bmatrix}\end{vmatrix}$.

27.曲线$(x(t),y(t),z(t))$在$(x,y,z)$点切线方向向量为$(x^{\prime},y^{\prime},z^{\prime})$.

28.曲面$f(x,y,z)=0$在$(x,y,z)$点法线方向向量为$(\frac{\partial f}{\partial x},\frac{\partial f}{\partial y},\frac{\partial f}{\partial z})$.

29.$\vec{a}=(x_{1},y_{1},z_{1})$,$\vec{b}=(x_{2},y_{2},z_{2})$方向相同$\Rightarrow$$\frac{x_{1}}{x_{2}}=\frac{y_{1}}{y_{2}}=\frac{z_{1}}{z_{2}}$$\nRightarrow$$\begin{cases}
x_{1}=x_{2}\\
y_{1}=y_{2}\\
z_{1}=z_{2}
\end{cases}$.

30.求函数在曲面上的最值(条件最值),因最值点一定在曲面上,可将曲面作为乘子引入,令$F=(\ln)f(x,y,z)+\lambda\cdot g(x,y,z)$(拉格朗日乘子法).

31.海伦公式:$S=\sqrt{\frac{L}{2}(\frac{L}{2}-a)(\frac{L}{2}-b)(\frac{L}{2}-c)}$.

32.$z(x,y)$在$(x_{0},y_{0})$处满足$\frac{\partial z}{\partial x}=\frac{\partial z}{\partial y}=0$,$A=\frac{\partial^{2}z}{\partial x^{2}},B=\frac{\partial^{2}z}{\partial x\partial y},C=\frac{\partial^{2}z}{\partial y^{2}}$,若:1.$B^{2}-AC<0$,
$1^{\circ}$.$A>0$,取极小值,$2^{\circ}$.$A<0$,取极大值;2.$B^{2}-AC>0$,非极值点.

33.对$f(x,y)$若$\frac{\partial^{2}f}{\partial x\partial y}=0$,则$\frac{\partial f}{\partial y}=\int0dx=\varphi(y)$,$f=\int\varphi(y)dy=\psi(y)+C(x)$,对变量$x$积分会出现$\varphi(y)$.

34.不要把法向量与$(\frac{\partial F}{\partial x},\frac{\partial F}{\partial y},\frac{\partial F}{\partial z})$划等号,它们是共线关系.

35.多重积分时,$x$的上下限为区域画横线,$y$的上下限为区域画竖线.
\end{document}

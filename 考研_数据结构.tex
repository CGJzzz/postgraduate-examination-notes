%% LyX 2.2.3 created this file.  For more info, see http://www.lyx.org/.
%% Do not edit unless you really know what you are doing.
\documentclass[UTF8]{ctexart}
\usepackage{amsmath}
\usepackage{fontspec}
\usepackage[unicode=true,pdfusetitle,
 bookmarks=true,bookmarksnumbered=true,bookmarksopen=true,bookmarksopenlevel=1,
 breaklinks=false,pdfborder={0 0 1},backref=false,colorlinks=false]
 {hyperref}

\makeatletter

%%%%%%%%%%%%%%%%%%%%%%%%%%%%%% LyX specific LaTeX commands.
%% Because html converters don't know tabularnewline
\providecommand{\tabularnewline}{\\}

%%%%%%%%%%%%%%%%%%%%%%%%%%%%%% User specified LaTeX commands.
% 重定义\nobreakspace命令,避免XeTeX出错(2.1.2版本后已不需要)
\DeclareRobustCommand\nobreakspace{\leavevmode\nobreak\ }

% 解决pdflatex不支持插入eps图片问题
\usepackage{epstopdf}
\usepackage{listings}
\usepackage{tikz}
\usepackage{tikz-qtree}
\usepackage[left=2cm, right=2cm, top=2cm]{geometry}
\usetikzlibrary{calc,shapes.multipart,chains,arrows,positioning}
\tikzset{
    squarecross/.style={
        draw, rectangle,minimum size=18pt, fill=orange!80,
        inner sep=0pt, text=black,
        path picture = {
            \draw[black]
            (path picture bounding box.north west) --
            (path picture bounding box.south east)
            (path picture bounding box.south west) --
            (path picture bounding box.north east);
        }
    }
}

\makeatother

\usepackage{xunicode}
\begin{document}
1.深度为h的平衡二叉(AVL)树的最少结点数为$N_{h}$($N_{h}$个结点的AVL树最大深度为h),则$N_{h}$和h满足:$N_{0}=0$,$N_{1}=1$,$N_{2}=2$;$N_{h}=N_{h-1}+N_{h-2}+1$。

2.顺序表设计算法对每个元素写出位移量(或目标索引)并观察规律。和微积分一样,不要纠结起点,应关注中间过程和一般情形。

3.折半查找要注意:\lstset{language=C} 
\begin{lstlisting} 
	while(low <= high)
	low = mid+1;
	high = mid - 1;
\end{lstlisting}

~~~~~~~~high后插入。

4.插入链表有尾插法和头插法两种。尾插法(表尾插入,q为表尾):\lstset{language=C} 
\begin{lstlisting} 
	q->next = p; q = q->next;
\end{lstlisting}

头插法(表头插入,q为表头):\lstset{language=C} 
\begin{lstlisting} 
	tmp = q->next; q->next = p; p->next = tmp;
\end{lstlisting}

5.\&会改变地址中的内容和地址本身,如\lstset{language=C} 
\begin{lstlisting} 
void change(Type &adress) {
	adress = xxx;
}
\end{lstlisting}

6.出栈序列一个元素右侧元素中值比它小(优先顺序在它前面)的为尚未出栈的元素,这些元素之间一定满足与优先顺序相反的顺序排序,如a<b<c(a先入栈),则cab不可能,因a<c,b<c,ab顺序与优先顺序相同,不符合要求。

7.循环队列队尾元素在队头元素之前时,因队尾指针指向队尾下一个元素位置,所以队头指针队尾指针指向同一位置,此时队空(同理也可使队头指针指向前一个增加头结点,队尾指针指队尾)。

8.输入受限的双端队列:$\leftarrow1\ 2\ 3\ 4\leftrightarrow$若第一个输出4,即(4 x x
x形式),则此时1,2,3一定还在队里,因此下一个输出的只能是1或3。因此,输入受限队列夹心的中心元素不能先输出,如123中2为夹心元素,因此4213中比4小的元素中有2夹心先出,不满足要求。输出受限输出序列不能有山峰,如4132中比4小的元素中3为山峰,不满足要求。

9.只有度为0和2的结点的二叉树最大深度为$\frac{1+n}{2}$,最少结点个数为2h+1。

10.满二叉树最底层结点数比其余结点数之和多1,第i层有$2^{i-1}$个结点。

11.因\Tree [.O [.O [.O  ] [.O  ] ] [.O [.O  ]  ] ]与\Tree [.O [.O [.O  ] [.O  ] ] [.O   ] ]叶结点数相同,完全二叉树最多结点数应在右图基础上加1。

12.n个结点的m叉树最小高度为$\lceil log_{m}(n(m-1)+1)\rceil$。

13.有n个结点的完全二叉树高度为$\lfloor log_{2}n\rfloor+1$。

14.先序遍历根和左儿子(或唯一儿子)相邻,右儿子为第一个非左子树结点。

15.如果一个名词能确定数据具体存储方式,则其为存储(物理)结构。

16.先序线索树前驱,后序线索树后继不能有效求解。

17.前序序列为入栈次序,中序序列为出栈次序。

18.后序非递归遍历二叉树:

\lstset{language=C} 
\begin{lstlisting} 
typedef struct{ BinaryTree data; int tag; } Stack;
void PostOrderPrint(BinaryTree t) {
	Stack s[MAX_SIZE]; int top = 0;
	BinaryTree p = t;
	while( p || top > 0) {
		if(p) { 
			s[++top].data = p; 
			s[top].tag = 0; 
			p = p->left; 
		}
		else {
			p = s[top].data;
			if(p->right && s[top].tag!=1) { 
				s[top].tag = 1; 
				p = p->right; 
			}
			else { 
				cout<<s[top--].data->data; 
				p = NULL; }
		}
	}
}
\end{lstlisting}

19.使用栈或队列的算法设计:a.考虑好工作栈,工作队列和工作指针的取值范围(空间,所有可能值),再从取值范围中划分出不同情况用作条件判断。如18中p应遍历所有结点的左右子树(包括叶),因此应考虑p指空(空子树)的情况。s应存放回溯时要访问的根结点且所有结点都应入栈一次,只访问从s中弹出的结点,因此入栈元素应为p第一次指向的结点,出栈条件应为左右子树都遍历过p指空时。b.考虑变量范围后再考虑变量取值顺序。18中出栈顺序为后序序列,入栈顺序应为先序序列。

20.简易栈:初始化:\lstset{language=C} 
\begin{lstlisting} 
ElemType stack[MAX_SIZE]; int top = 0;
\end{lstlisting}

入栈:\lstset{language=C} 
\begin{lstlisting} 
stack[++top]=element;
\end{lstlisting}

出栈:\lstset{language=C} 
\begin{lstlisting} 
element = stack[top--];
\end{lstlisting}

栈空:\lstset{language=C} 
\begin{lstlisting} 
top == 0
\end{lstlisting}

栈顶:\lstset{language=C} 
\begin{lstlisting} 
element = stack[top];
\end{lstlisting}

简易队列:初始化:\lstset{language=C} 
\begin{lstlisting} 
ElemType queue[MAX_SIZE]; int front = 0, rear = 0;
\end{lstlisting}

入队:\lstset{language=C} 
\begin{lstlisting} 
queue[rear++] = element;
\end{lstlisting}

出队:\lstset{language=C} 
\begin{lstlisting} 
element = queue[front++];
\end{lstlisting}

队空:\lstset{language=C} 
\begin{lstlisting} 
front == rear
\end{lstlisting}

队头/尾:\lstset{language=C} 
\begin{lstlisting} 
element1 = queue[front];
element2 = queue[rear];
\end{lstlisting}

21.二叉排序树插入时只插结点处。

22.平衡因子绝对值大于等于1的结点互换高度,一正一负中间插,中间子树根转移。

23.平衡二叉树取最少结点数即非叶结点平衡因子均为1的情况。

24.由哈夫曼树的构造过程可得哈夫曼树有n个叶结点,n-1个非叶结点,无度为1的结点。

25.前缀码所有编码结点均为叶结点,没有一个编码是另一个编码的前缀。

26.叶结点权值组可能对应多个哈夫曼树(带权路径相同但形状不同)。

27.\Tree [.O [.O [.xxxx ] [.O [.x ] [.x ]]]  [.x ]]其中xxxx为省略的子树,x为查找失败处。查找失败要进行三次比较,因此查找失败的平均查找长度为$\frac{...+3\times2}{...+2}$,其中2为两个失败结点(x)。

28.堆是一个完全二叉树。

29.中序非递归遍历二叉树:

\lstset{language=C} 
\begin{lstlisting} 
void InOrderPrint(BinaryTree t) {
	BinaryTree stack[MAX_SIZE]; int top = 0;
	BinaryTree p = t;
	while(p || top>0) {
		if(p) { 
			stack[++top] = p; 
			p = p->left; 
		}
		else { 
			p = stack[top--]; 
			cout<<p->data; 
			p = p->right; 
		}
	}
}
\end{lstlisting}

30.树的根到叶结点的路径长度=深度-1

31.合并长度为m和n的有序表最多需m+n-1次比较(大小大小...)

32.图的遍历就是从一个结点出发遍访其余结点。错误!非连通图无法实现。

33.非连通图可以有单个度为0的结点,不占用任何一条边。

34.边数最少的连通图和边数最少的强连通图都是简单回路。

35.n个顶点无向图,要有$\frac{(n-1)(n-2)}{2}\text{(完全图)}+1$个边以确保其为连通图(完全图加一边)。

36.极大连通子图=连通分量,极小连通子图=最小生成树(MST)

37.当某个顶点只有出弧没有入弧时,其他顶点无法到达这个顶点,不可能与其他顶点和边构成强连通分量(这个顶点为一个单独强连通分量)。

38.顶点(事件)$v_{k}$的最早发生时间$ve(k)$为从源点到$v_{k}$最长路径(取最大值)。

顶点(事件)$v_{k}$的最迟发生时间$vl(k)$为从源点到汇点最长路径长度减去从$v_{k}$到汇点最长路径长度(取最小值)。

边(弧,活动)$a_{i}$的最早开始时间$e(i)$为从源点到$a_{i}$的尾顶点(活动起点)的最长路径。

边(弧,活动)$a_{i}$的最迟开始时间$l(i)$为从源点到汇点的最长路径长度减去从$a_{i}$的头顶点(活动终点)路径长度再减去$a_{i}$权,即$l(i)=vl(k)-weight(a_{i})$,其中$a_{i}=\vec{v_{j}v_{k}}$。

若$ve(k)=vl(k)$,则$v_{k}$为关键路径上的点。

39.Prim算法:新边连旧边,无环。Kruskal算法:最小边,无环。

40.Dijkstra算法:a.选择离出发点最近的点,确定这个点的最短路径。b.修改其它点的最短路径(广度优先)。

41.带权有向图邻接表:\begin{tikzpicture}[         
	list/.style={              
			rectangle split,draw,rectangle split parts=3,rectangle split horizontal, text=black,      		},         
		->, start chain 
	]

\node[list,on chain] (A) {\nodepart{second} vi};   
\node[list,on chain] (B) {\nodepart{second} weight};   
\node[list,on chain] (C) {\nodepart{second} ...};

\path[*->] let \p1 = (A.three), \p2 =  (A.two) in (\x1,\y2) edge [bend left] ($(B.one)+(0,0.2)$);   
\path[*->] let \p1 = (B.three), \p2 = (B.center) in (\x1,\y2) edge [bend left] ($(C.one)+(0,0.2)$);  
\end{tikzpicture}

42.AOE网一定从一个事件(顶点)出发止于一个事件(汇点)。

43.%
\begin{tabular}{|c|c|c|}
\hline 
算法 & 图的类型 & 算法结果\tabularnewline
\hline 
\hline 
Prim,Kruskal & 带权无向连通图 & 最小生成树\tabularnewline
\hline 
Dijkstra,Floyd & 带权有向图 & 最短路径\tabularnewline
\hline 
\end{tabular}

44.折半查找比较次数 例:1 2 3 4 5 (找4),$\frac{1+5}{2}=3$,$\frac{4+5}{2}=4$,因此共比较两次。

45.B+树:非失败,非根结点关键字个数n满足$\lceil\frac{m}{2}\rceil\leq n\leq m$;根结点关键字个数满足$1\leq n\leq m$

B树:非失败,非根结点关键字个数n满足$\lceil\frac{m}{2}\rceil-1\leq n\leq m-1$,子树个数$\lceil\frac{m}{2}\rceil\leq n_{h}\leq m$;根结点关键字个数满足$1\leq n\leq m-1$,子树个数$2\leq n_{h}\leq m$

m阶关键字(非叶结点)个数为n的B树,高度h满足$log_{m}(n+1)\leq h\leq log_{\lceil\frac{m}{2}\rceil}(\frac{n+1}{2})+1$,失败结点个数为n+1。

m阶高为h的B树关键字最多为$m^{h}-1$

B树失败结点为空,不占高度,插入访问失败结点,插在叶结点上。

B树插入溢出时中间结点上游父结点并增加分支。

B树删除时缺的位置可由同一棵子树根上的关键字向下滑动填上,若下滑后父结点关键字过少,则借兄弟或父结点反方向下潜。

46.表项:散列表中的记录个数。

47.查找失败时最后还要和空元素比较一次。要用散列函数值域来计算查找失败平均查找长度,成功用元素个数算。(成个失域)

48.KMP算法Next数组求法:

a. next{[}1{]}=0,next{[}2{]}=1

b. 从当前计算的元素之前截字符串,根据左侧next值判断从几位开始比较,若成功+1,失败-1,直到成功或减到1为止。

c. 前n位与倒数后n位比较。

d. next数组输入子串匹配失败的位置,输出要滑动到的位置。如a b c a c 在j=5(c)匹配失败,由next{[}5{]}=2,应从b开始匹配。

e. 统一next{[}{]} sstring{[}{]}的下标,都从0开始时next{[}0{]}=-1

49.对任意的n个关键字的序列进行基于比较的排序,至少要进行$\lceil log_{2}(n!)\rceil$次两两比较。

50.除了直插,冒泡,归并,基数外都不稳定。(直基冒归)

51.第i趟快排,冒泡,选择后,会有$\geq i$个元素在最终位置。

52.涉及稳定性时,哪个数据项要求全局有序后排哪个(保证有序)。

53.堆的删除要将堆尾元素置顶,一次下游最多比较2次。

54.k路归并,排序趟数m满足$k^{m}\geq N$,$m=\lceil log_{k}N\rceil$

55.m叉哈夫曼树,度0结点数为$n_{0}$,度为m结点数为$n_{m}$,则有:$n_{m}=\frac{n_{0}-1}{m-1}$多余结点数$u=(n_{0}-1)\%(m-1)$,应增加空归并段数为$m-u-1$

56.并行m路归并要设置2m个输入缓冲区,2个输出缓冲区。

57.Hash函数 H(key)=key\%p,N为表长,n为关键字个数,$\alpha=\frac{n}{N}$为填装因子,p应取不大于表长的最大素数,$N=\lceil\frac{n}{\alpha}\rceil$链地址法(即链表)$\neq$线性探测再散列法。

58.Floyd算法$A^{(-1)}[i][j]=arcs[i][j]$,$A^{(k)}[i][j]=Min\{A^{(k-1)}[i][j],A^{(k-1)}[i][k]+A^{(k-1)}[k][j]\}$,$k=0,1,\cdots,n-1$。$A_{n\times n}^{(k)}=Min\{A_{n\times n}^{(k-1)},\begin{bmatrix}a_{1k}\\
\cdots\\
a_{nk}
\end{bmatrix}\oplus\begin{bmatrix}a_{k1} & \cdots & a_{kn}\end{bmatrix}\}$,其中$\oplus$表示将矩阵乘法中元素间的运算由乘改为加。Min\{\}表示将矩阵对应元素取最小值作为新矩阵值。

59.邻接表:第i结点的第j邻接点:\lstset{language=C} 
\begin{lstlisting} 
g.vertices[g.vertices[i]->first->...(->next (j-1 nexts))->adjvex]
\end{lstlisting}

对应网的权Aij:\lstset{language=C} 
\begin{lstlisting} 
g.vertices[i]->first(->next(j nexts))->info
\end{lstlisting}

60.平均查找长度$\neq$最大查找长度,看清“平均”“最坏情况”。

61.有向图$v_{i}$到$v_{j}$所有路径:令visited{[}j{]}=true,从$v_{i}$对图进行深度优先遍历,若某次NextAdj操作得到$v_{j}$,则逆序输出(从stack{[}0{]}开始)栈和$v_{j}$。

62.邻接表深度优先非递归遍历:

\lstset{language=C} 
\begin{lstlisting} 
void DFS(ALGraph &g,int vi){     
	//从vi顶点开始进行深度优先遍历     
	ArcNode *stack[g.arcnum];     
	int top=0;     
	bool visited[g.vexnum];     
	for(int i=0; i < g.vexnum; i++) 
		visited[i]=false;     
	ArcNode *p=g.vertices[vi]->firstarc;     
	cout<<vi<<"  ";     
	visited[vi]=true;         
	while(p||top>0) {         
		if(p) {             
			if(!visited[p->adjvex]) {                 
				cout<<p->adjvex<<"  ";                 
				visited[p->adjvex]=true;                 
				stack[++top]=p;                 
				p=g.vertices[p->adjvex]->firstarc;             
			}             
			else                 
				p = p->nextarc;         
			}         
			else             
				p = stack[top--]->nextarc;     
		}     
		cout<<endl; 
}
\end{lstlisting}

63.基于交换排序思想:low指向左侧第一个要交换的,high指向右侧第一个要交换的,当low>=high时结束,若使用A{[}0{]}保存A{[}low{]}第一次交换改为\lstset{language=C} 
\begin{lstlisting} 
A[low]=A[high];
\end{lstlisting},与之对应的交换改为A{[}high{]}=A{[}low{]},最后恢复A{[}low{]}可减少空间使用。

64.顶点k的偏心度为所有其余点到k最短路径中的最大值,最小偏心度的顶点为中心点。

65.Warshall算法判断i到j是否有一条路径:$A_{ij}^{(k+1)}=A_{ij}^{(k)}\cup(A_{ik}^{(k)}\cap A_{kj}^{(k)})$

66.基数排序=队列+散列

67.每个判断条件下都应改变状态以避免死循环(输出不算改变状态,赋值算)。

68.算法中的循环往往对应一个数学归纳过程。

69.题目中非形式的数据结构要用伪码形式化的定义出来。

70.最后不要忘了写算法复杂度。
\end{document}
